\documentclass{beamer}%linha 527 e 540
\usepackage{atalhos,juniorbeamer}
\setbeamertemplate{theorems}[numbered] %para numerar as definições de teoremas
\newtheorem{defin}{Formulário}[section]
\newtheorem{definicao}[defin]{Definição}
\everymath{\displaystyle }
\usepackage{pgfplots,tikz,comment}
\def\azul{\color{blue!90!black}}
\pgfplotsset{width=7cm}
\usetikzlibrary{arrows}
\setlength{\parskip}{0.6em}
\AtBeginSection[]
{
   \begin{frame}
        \frametitle{Conteúdo}
        \tableofcontents[currentsection,currentsubsection]
   \end{frame}
}
\renewcommand{\familydefault}{\rmdefault}
\newtheorem{teo}[defin]{Teorema}
\mode<presentation> {
% The Beamer class comes with a number of default slide themes
% which change the colors and layouts of slides. Below this is a list
% of all the themes, uncomment each in turn to see what they look like.
%\usetheme{default}
%\usetheme{AnnArbor}
%\usetheme{Antibes}
%\usetheme{Bergen}
%\usetheme{Berkeley}
%\usetheme{Berlin}
%\usetheme{Boadilla}
%\usetheme{CambridgeUS}
%\usetheme{Copenhagen}
%\usetheme{Darmstadt}%
%\usetheme{Dresden}
%\usetheme{Frankfurt}%
%\usetheme{Goettingen}
%\usetheme{Hannover}
\usetheme{Ilmenau}%%
%\usetheme{JuanLesPins}
%\usetheme{Luebeck}
%\usetheme{Madrid}
%\usetheme{Malmoe}
%\usetheme{Marburg}
%\usetheme{Montpellier}
%\usetheme{PaloAlto}
%\usetheme{Pittsburgh}
%\usetheme{Rochester}
%\usetheme{Singapore}
%\usetheme{Szeged}
%\usetheme{Warsaw}
% As well as themes, the Beamer class has a number of color themes
% for any slide theme. Uncomment each of these in turn to see how it
% changes the colors of your current slide theme.
%\usecolortheme{albatross}
%\usecolortheme{beaver}
%\usecolortheme{beetle}
%\usecolortheme{crane}
%\usecolortheme{dolphin}
%\usecolortheme{dove}
%\usecolortheme{fly}
%\usecolortheme{lily}
%\usecolortheme{orchid}
%\usecolortheme{rose}
%\usecolortheme{seagull}
%\usecolortheme{seahorse}
%\usecolortheme{whale}
%\usecolortheme{wolverine}
%\setbeamertemplate{footline} % To remove the footer line in all slides uncomment this line
\setbeamertemplate{footline}[page number] % To replace the footer line in all slides with a simple slide count uncomment this line
\setbeamertemplate{navigation symbols}{} % To remove the navigation symbols from the bottom of all slides uncomment this line
}
\newcommand{\enf}[1]{\textbf{\color{red} #1}}
\newcommand{\pag}{
	{
	\small[\insertframenumber{}/\inserttotalframenumber]
	}\xspace
}

\usepackage{txfonts,bclogo}
\usepackage{booktabs} % Allows the use of \toprule, \midrule and \bottomrule in tables

%----------------------------------------------------------------------------------------
%	TITLE PAGE
%----------------------------------------------------------------------------------------

\title[Estruturas de Controle]
{Programação em linguagem estruturada: estruturas de controle} % The short title appears at the bottom of every slide, the full title is only on the title page

\author{\bf Junior Rodrigues Ribeiro} % Your name
\institute[ICMC] % Your institution as it will appear on the bottom of every slide, may be shorthand to save space
{
USP --- ICMC \\ % Your institution for the title page
}
\date{\today} % Date, can be changed to a custom date
\setbeamertemplate{navigation symbols}{}
\date{\small 06 de Fevereiro de 2020 \\[.3cm] E-mail: \texttt{jrodrib@usp.br}}

%%%%%%%%%%%%%%%%%%%%%%%%%%%%%%%%%%%%%%%%%%%%%%%%%%%
%%%%%%%%%%%%%       INICIO       %%%%%%%%%%%%%%%%%%
%%%%%%%%%%%%%%%%%%%%%%%%%%%%%%%%%%%%%%%%%%%%%%%%%%%
%%%%%%%%%%%%%%%%%%%%%%%%%%%%%%%%%%%%%%%%%%%%%%%%%%%
\def\w{\omega}
\renewcommand{\familydefault}{\rmdefault}

\begin{document}
%%%%%%%%%%%%%%%%%%%%%%%%%%%%%%%%%%%% FRAME 0.0 %%%%
\begin{frame}
\titlepage
\end{frame}
%%%%%%%%%%%%%%%%%%%%%%%%%%%%%%%%%%%% FRAME 0.1 %%%%


%%%%%%%%%%%%%%%%%%%%%%%%%%%%%%%%%%%%%%%%%%%%%%%%%%%
%%%%%%%%%%%%%%%%     INICIO     %%%%%%%%%%%%%%%%%%%
%%%%%%%%%%%%%%%%%%%%%%%%%%%%%%%%%%%%%%%%%%%%%%%%%%%


\section[Estruturas]{Estruturas de controle}
\begin{frame}{Estruturas de controle}
São três as estruturas de controle que modelam um programa estruturado:
\begin{itemize}
\item Sequência (\textbf{vá para (goto)*});
\item Decisão (se-então-senão, interruptor-caso-padrão);
\item Repetição (faça-enquanto, enquanto, para);
\end{itemize}
Todas elas são importantes e são usadas com muita frequência.

%Há também comandos de desvio do fluxo de execução. São eles \textit{retorne, pare} e  \textit{continue}. São usados para modificar a sequência da execução e são usados muitas vezes em conjunto com estruturas de seleção.
\end{frame}



\section{Sequência}
\begin{frame}{Sequência}
As estruturas de sequência, como o próprio nome diz, indicam que o
programa será executado linha após linha, na mesma sequência em que
os comandos aparecem.

Antigamente, na programação não estruturada (antecessora da estruturada) havia a possibilidade de declarar seções de código explicitamente, e então mudar o curso de execução do programa com o \textbf{vá para (goto)}. Essa prática tornava os programas muito difíceis de ler, e por isso é incisivamente desencorajada.

O próximo exemplo ilustra como o \textit{goto} é problemático.
\end{frame}



\begin{frame}{Uso do \textbf{goto}: dificulta a leitura do código pelo programador}{Não é recomendado usá-lo!}
\begin{quote}\small
!programa para calcular soma = $\textstyle\sum_{k=1}^{5}k$\\
contador = 0\\
soma = 0\\
\textbf{goto} seção2\\
\textbf{seção1:}\\
escreva (soma)\qquad\qquad\qquad\qquad\!\!\!\! \fbox{15}\\
escreva ("fim do programa") \qquad \fbox{fim do programa}\\
\textbf{goto} seçãoFim\\
\textbf{seção2:}\\
contador = contador + 1\\
soma = soma + contador\\
se(contador < 5) \textbf{goto} seção2\\
senão \textbf{goto} seção1\\
\textbf{seçãoFim:}\\
escreva("Fim :)")\qquad\qquad\qquad\quad \fbox{\rm Fim :)}
\end{quote}
\end{frame}




\section[Seleção]{Decisão/Seleção/Condicional}
\begin{frame}{Decisão/Seleção}
As estruturas de decisão modificam o fluxo de execução, de acordo
com o valor lógico de uma expressão lógica avaliada.
\begin{itemize}\justifying
\item O exemplo mais
notório é o \textit{se-então-senão} (if-then-else).

{\azul se (expressão lógica) é verdadeira, então execute\\
\qquad    (sequência)\\
senão*, execute\\
\qquad    (sequência)\\
fim se}

\item Outra estrutura muito importante é o \textit{interruptor-caso-padrão}
(switch-case-default), que é uma sequência de comparações,
dividindo o fluxo de execução em uma porção de outros fluxos,
conforme cada caso. Uma determinada variável é selecionada e seu
valor é comparado com cada caso.
\end{itemize}

\end{frame}




\begin{frame}
interruptor-caso-padrão (switch-case-default)

{\azul interruptor (variável):\\
\qquad    caso valha x,\\
\qquad\qquad    (sequência)\\
\qquad    caso valha y,\\
\qquad\qquad    (sequência)\\
\qquad    caso valha z,\\
\qquad\qquad    (sequência)\\
\qquad    caso padrão*,\\
\qquad\qquad    (sequência)\\
fim interruptor
}

Os itens \textbf{senão} (else) e \textbf{caso padrão} (default) não são
obrigatórios, e podem ser omitidos.
\end{frame}






\section[Repetição]{Repetição/Looping/Laços/Iteração}
\begin{frame}
As estruturas de repetição geralmente combinam uma expressão lógica
e uma sequência de comandos. São elas: \textit{para},
\textit{enquanto} e \textit{faça-enquanto} (for, while, do-while).
\begin{itemize}\justifying
\item Para: é uma estrutura utilizada para quando o número de
repetições é conhecido. Neste caso, as repetições são realizadas
até que o número seja atingido. Outras implementações trabalham com
estruturas de dados iteráveis, de forma que a estrutura "para"\ é
repetida uma vez para cada elemento da estrutura de dados, até que ela
seja esgotada.


{\azul para x de 1 até 10, execute\\
\qquad  (sequência)\\
fim para
}


ou


{\azul para aluno em listaDeAlunos,\\
\qquad  (sequência)\\
fim para
}

\end{itemize}
\end{frame}



\begin{frame}
\begin{itemize}\justifying
\item Enquanto: a estrutura "enquanto"\ é utilizada sempre que não se
sabe a priori quantas vezes precisamos da repetição. Uma condição
lógica é útil para que a repetição termine em algum momento. Do
contrário, o programa executará para sempre. Assim, nessa estrutura,
precisamos de uma condição de parada, ou seja, enquanto a expressão
lógica for verdadeira, a sequência é executada.

{\azul
enquanto (expressão lógica) for verdadeira, execute\\
\qquad (sequência)\\
fim enquanto
}

Essa estrutura avalia primeiro a expressão lógica para então dar
sequência. Caso a expressão lógica já seja inicialmente falsa, a
sequência não será executada.
\end{itemize}
\end{frame}





\begin{frame}
\begin{itemize}\justifying
\item Faça-enquanto: a estrutura "faça-enquanto"\ só tem uma diferença em relação à
estrutura "enquanto": ela executa a sequência de instruções para
depois verificar a expressão lógica (a sequência é
executada pelo menos uma vez). É frequentemente usada em menus.

{\azul
faça\\
\qquad (sequência)\\
enquanto (expressão lógica) for verdadeira
}
\end{itemize}
\end{frame}

\section[Aninhamento]{Aninhamento de estruturas (nesting)}
\begin{frame}{Aninhamento de estruturas (nesting)}
As estruturas apresentadas podem também ser usadas repetidas vezes e
combinadas. Um bom exemplo é escrever uma matriz (3x5) na tela.

{\azul
para linha de 1 até 3,\\
\qquad para coluna de 1 até 5,\\
\qquad\qquad escreva( matriz[linha][coluna] + \_espaço)\\
\qquad fim para\\
\qquad escreva(\_quebra\_de\_linha)\\
fim para
}

<Teste de mesa>
\end{frame}



\begin{frame}
Outro exemplo: escrever o nome de alunos aprovados

{\azul
para aluno em listaDeAlunos,\\
\qquad se aluno.médiaFinal >= 5.0,\\
\qquad\qquad escreva( aluno.nome + " está aprovado.")\\
\qquad senão, se aluno.médiaFinal >= 3.0\\
\qquad\qquad\qquad escreva( aluno.nome + " está de recuperação")\\
\qquad\qquad senão\\
\qquad\qquad\qquad escreva( aluno.nome + " está reprovado")\\
\qquad\qquad fim se\\
\qquad fim se\\
\qquad escreva(\_quebra\_de\_linha)\\
fim para
}
\end{frame}



\begin{frame}
Outro exemplo: menu de opções

{\azul
faça\\
\qquad escreva("1 - listar produtos")\\
\qquad escreva("2 - cadastrar produtos")\\
\qquad escreva("3 - editar produtos")\\
\qquad escreva("4 - deletar produtos")\\
\qquad escreva("0 - sair")\\
\qquad opção = leia( )\\
\qquad interruptor(opção)\\
\qquad\qquad caso 1\\
\qquad\qquad\qquad  listarProdutos( )\\
\qquad\qquad caso 2\\
\qquad\qquad\qquad  cadastrarProdutos( )
}
\end{frame}



\begin{frame}\
{\azul
\qquad\qquad caso 3\\
\qquad\qquad\qquad  editarProdutos( )\\
\qquad\qquad caso 4\\
\qquad\qquad\qquad  deletarProdutos( )\\
\qquad\qquad caso 0\\
\qquad\qquad\qquad escreva("Você escolheu sair. Até breve!")\\
\qquad\qquad padrão\\
\qquad\qquad\qquad escreva("Opção inválida.")\\
\qquad fim interruptor\\
enquanto(opção != 0)
}
\end{frame}




\section{Desvios}
\begin{frame}{Estruturas de desvios}
Há também a possibilidade de desviar o fluxo de execução de um
programa, usando \textit{tentar-pegar} (try-catch), \textit{retornar}
(return), \textit{parar} (break) e  \textit{continuar} (continue).

\begin{itemize}\justifying
\item Tentar-Pegar: é uma estrutura parecida com Se-Então-Senão, mas
para tratamento de erros. \\
{\azul
tentar\\
\qquad (sequência)\\
pegar (erro)\\
\qquad (sequência para tratar o erro)\\
fim tentar
}

Exemplo: Tentar abrir um arquivo. Se o arquivo não existir ocorrerá
um erro. Algo que pode ser feito é dizer ao usuário que o arquivo
não existe.
\end{itemize}

\end{frame}



\begin{frame}
 \begin{itemize}\justifying\small
\item Retornar: é usado dentro de funções. Diz ao programa que não
há interesse de continuar a execução, mas sim, de retornar à linha
na qual a função foi chamada (usando um se-senão?). Existe a
possibilidade de carregar consigo alguma informação para o código que a
chamou. Em algumas linguagens, esta estrutura é implícita ao final da
função, não precisando ser declarada.\\
{\azul
função cadastro(parâmetro1, parâmetro2)\\
\qquad (sequência)\\
\qquad se (expressão lógica)\\
\qquad\qquad retorne 5\\
\qquad senão, se (expressão lógica)\\
\qquad\qquad\qquad retorne -5\\
\qquad\qquad senão\\
\qquad\qquad\qquad (sequência)\\
\qquad\qquad fim se\\
\qquad fim se\\
\qquad retorne 0\\
fim função\\
\\
código = cadastro("João", "Maria")
}
\end{itemize}

\end{frame}



\begin{frame}
 \begin{itemize}\justifying
\item Parar: é usado dentro de repetições. Diz ao programa que não
há interesse de continuar as repetições (usando um se-senão?). O
fluxo de execução é enviado ao final do laço.\\
{\azul
enquanto (expressão lógica)\\
\qquad (sequência)\\
\qquad se (expressão lógica)\\
\qquad\qquad pare >>\\
\qquad fim se\\
\qquad (sequência)\\
fim enquanto>>\\
(sequência)
}
\end{itemize}

\end{frame}


\begin{frame}
 \begin{itemize}\justifying
\item Continuar: também é usado dentro de repetições. Diz ao programa que não
há interesse de continuar aquela repetição em específico (usando um se-senão?).
O fluxo de execução é enviado ao começo do laço, pronto para a
iteração seguinte.\\
{\azul
>> enquanto (expressão lógica)\\
\qquad (sequência)\\
\qquad se (expressão lógica)\\
\qquad\qquad continue >>\\
\qquad fim se\\
\qquad (sequência)\\
fim enquanto\\
(sequência)
}
\end{itemize}

\end{frame}




%%%%%%%%%%%%%%%%%%%%%%%%%%%%%%%%%%%%%%%%%%%%%%%%%%%%
%%%%%%%%%%%%%%%%%     REFERÊNCIAS    %%%%%%%%%%%%%%%
%%%%%%%%%%%%%%%%%%%%%%%%%%%%%%%%%%%%%%%%%%%%%%%%%%%%
\section{Referências}
\begin{frame}{Referências}
\footnotesize{
\begin{thebibliography}{99}
\bibitem[Martin, 1991]{martin} MARTIN, J. Técnicas estruturadas e
CASE. Obra traduzida. São Paulo. Makron Books, 1991.

\bibitem[Python.Org]{python} Linguagem Python
\url{https://docs.python.org/pt-br}

\bibitem[CPlusPlus]{cpp} Linguagem C/C++
\url{http://www.cplusplus.com}
\end{thebibliography}
}
\end{frame}








\begin{frame}
\Huge{\centerline{\bcnote \dotfill \bcclesol \dotfill$_{\bfF\ \bfI\ \bfM}$\dotfill \bcclefa \dotfill \bcnote}}

\begin{center}
\bctrefle Muito obrigado!\bctrefle
\end{center}
\end{frame}
\end{document}
